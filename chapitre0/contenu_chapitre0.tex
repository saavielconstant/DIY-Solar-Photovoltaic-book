\input{lien_header}
\begin{document}
\addcontentsline{toc}{chapter}{Introduction}
\chapter*{Introduction}
Ces dernières années, la question de l’énergie est devenue une préoccupation mondiale au niveau de tous les maillons et secteurs des sociétés. La raison est simple : Les ressources pétrolières, principales sources d’énergie jusque la, s’épuisent a une vitesse phénoménale et les réserves restantes, ne donnent aucun espoir d’accès a l’énergie dans un futur proche. Nul besoin donc de vous dire que si des mesures adéquates ne sont pas prises, le monde pourrait très rapidement être confronté à de sérieux problèmes d’énergies. C’est pour- quoi ces dernières années, une course vers le développement de nouvelles alternatives aux ressources pétrolières est engage dans le monde entier aussi bien dans le milieu de la recherche que dans le milieu industriel.


Une alternative intéressante qui est activement explorée actuellement est le solaire photovoltaïque pour plus d’une raison :

\begin{itemize}
\item Le gisement en matière de ressource solaire est disponible en abondance et sur toute l’étendu du  globe terrestre.
\item	Il offre beaucoup plus de flexibilité que les autres alternatives, permettant d’alimenter en électricité toute sorte de local, en zone urbaine ou isole, et surtout offrant la possibilité a toute personne de produire sa propre électricité, voir même la revendre.
\item	La conception et l’installation pour des petits locaux n’a rien d’extraordinaire et peut être réalisée par tous.
\end{itemize}
Avoir son propre système photovoltaïque présente énormément d’avantages : autonomie, réduction du coût d’électricité, retour d’investissement sur le long terme et j’en passe. Cependant, bien qu’a la portée de tous de part sa simplicité de déploiement a petite échelle, il n’est pas exclu que l’installation soit mal conçue et que l’on fasse de grandes pertes financières pour aboutir a une production quasiment insatisfaisante, voir inutile.


L’objectif de ce livre est double :

\begin{itemize}
 \item Encourager et persuader toute personne toujours dans le doute à cause de la crainte de l’échec, qu’elle peut avoir sa propre installation photovoltaïque pleinement fonctionnelle en suivant les conseils et astuces simples présentées ici.
 \item Mettre à la disposition du novice intéressé par l’installation de son propre système photovoltaïque un guide pratique et simple expliquant pas à pas la procédure sur la base d’exemples qu’il suffira de suivre pour déployer son propre système.
\end{itemize}

Le livre s’articule autour de chapitres aux travers desquels vous apprendrez a réaliser les taches essentielles a l’installation d’un système photovoltaïque en 12V DC pour vos bateaux, camping cars, fourgonnettes, cabanes, et petites maison.
\begin{itemize}
\item	Dans les deux premiers chapitres, vous apprendrez les bases de l’électricité et les caractéristiques essentielles des éléments constitutifs d’un système de production photovoltaïque autonome.
\item	Dans les deux chapitres suivants, vous allez pouvoir estimer vos besoins journaliers en énergie électrique puis déduire le nombre de panneaux solaires et de batteries de stockage dont vous aurez besoins. Vous apprendrez également à choisir judicieusement la section du câble de raccordement selon le courant maximum qui va y passer.
\item	Vous apprendrez ensuite a travers d’autres chapitres comment connecter les différents éléments ensemble pour atteindre la puissance désirée
\item	Les derniers chapitres concerneront la création d’un circuit 12 volts pour alimenter des appareils électriques et la maintenance du système.
\end{itemize}
\end{document}
